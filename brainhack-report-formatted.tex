%% BioMed_Central_Tex_Template_v1.06
%%                                      %
%  bmc_article.tex            ver: 1.06 %
%                                       %

%%IMPORTANT: do not delete the first line of this template
%%It must be present to enable the BMC Submission system to
%%recognise this template!!

%%%%%%%%%%%%%%%%%%%%%%%%%%%%%%%%%%%%%%%%%
%%                                     %%
%%  LaTeX template for BioMed Central  %%
%%     journal article submissions     %%
%%                                     %%
%%          <8 June 2012>              %%
%%                                     %%
%%                                     %%
%%%%%%%%%%%%%%%%%%%%%%%%%%%%%%%%%%%%%%%%%


%%%%%%%%%%%%%%%%%%%%%%%%%%%%%%%%%%%%%%%%%%%%%%%%%%%%%%%%%%%%%%%%%%%%%
%%                                                                 %%
%% For instructions on how to fill out this Tex template           %%
%% document please refer to Readme.html and the instructions for   %%
%% authors page on the biomed central website                      %%
%% http://www.biomedcentral.com/info/authors/                      %%
%%                                                                 %%
%% Please do not use \input{...} to include other tex files.       %%
%% Submit your LaTeX manuscript as one .tex document.              %%
%%                                                                 %%
%% All additional figures and files should be attached             %%
%% separately and not embedded in the \TeX\ document itself.       %%
%%                                                                 %%
%% BioMed Central currently use the MikTex distribution of         %%
%% TeX for Windows) of TeX and LaTeX.  This is available from      %%
%% http://www.miktex.org                                           %%
%%                                                                 %%
%%%%%%%%%%%%%%%%%%%%%%%%%%%%%%%%%%%%%%%%%%%%%%%%%%%%%%%%%%%%%%%%%%%%%

%%% additional documentclass options:
%  [doublespacing]
%  [linenumbers]   - put the line numbers on margins

%%% loading packages, author definitions

\documentclass[twocolumn]{bmcart}% uncomment this for twocolumn layout and comment line below
%\documentclass{bmcart}

%%% Load packages
\usepackage{amsthm,amsmath}
\usepackage{siunitx}
\usepackage{mfirstuc}
%\RequirePackage{natbib}
\usepackage[colorinlistoftodos]{todonotes}
\RequirePackage{hyperref}
\usepackage[utf8]{inputenc} %unicode support
%\usepackage[applemac]{inputenc} %applemac support if unicode package fails
%\usepackage[latin1]{inputenc} %UNIX support if unicode package fails
\usepackage[htt]{hyphenat}

\usepackage{array}
\newcolumntype{L}[1]{>{\raggedright\let\newline\\\arraybackslash\hspace{0pt}}p{#1}}
\providecommand{\tightlist}{%
  \setlength{\itemsep}{0pt}\setlength{\parskip}{0pt}}
  
%%%%%%%%%%%%%%%%%%%%%%%%%%%%%%%%%%%%%%%%%%%%%%%%%
%%                                             %%
%%  If you wish to display your graphics for   %%
%%  your own use using includegraphic or       %%
%%  includegraphics, then comment out the      %%
%%  following two lines of code.               %%
%%  NB: These line *must* be included when     %%
%%  submitting to BMC.                         %%
%%  All figure files must be submitted as      %%
%%  separate graphics through the BMC          %%
%%  submission process, not included in the    %%
%%  submitted article.                         %%
%%                                             %%
%%%%%%%%%%%%%%%%%%%%%%%%%%%%%%%%%%%%%%%%%%%%%%%%%


%\def\includegraphic{}
%\def\includegraphics{}

%%% Put your definitions there:
\startlocaldefs
\endlocaldefs


%%% Begin ...
\begin{document}

%%% Start of article front matter
\begin{frontmatter}

\begin{fmbox}
\dochead{Report from 2015 OHBM Hackathon (HI)}

%%%%%%%%%%%%%%%%%%%%%%%%%%%%%%%%%%%%%%%%%%%%%%
%%                                          %%
%% Enter the title of your article here     %%
%%                                          %%
%%%%%%%%%%%%%%%%%%%%%%%%%%%%%%%%%%%%%%%%%%%%%%

\title{Distributed collaboration: the case for the enhancement of Brainspell's
interface}
\vskip2ex
\projectURL{Project URL: \url{http://github.com/r03ert0/brainspell-brainhack}}

\author[
addressref={aff1, aff2},
%
email={rto@pasteur.fr}
]{\inits{APB} \fnm{AmanPreet} \snm{Badhwar}}
\author[
addressref={aff3},
%
email={rto@pasteur.fr}
]{\inits{DNK} \fnm{David} \snm{Kennedy}}
\author[
addressref={aff4},
%
email={rto@pasteur.fr}
]{\inits{DNK} \fnm{Jean-Baptiste} \snm{Poline}}
\author[
addressref={aff5},
corref={aff5},
email={rto@pasteur.fr}
]{\inits{RT} \fnm{Roberto} \snm{Toro}}

%%%%%%%%%%%%%%%%%%%%%%%%%%%%%%%%%%%%%%%%%%%%%%
%%                                          %%
%% Enter the authors' addresses here        %%
%%                                          %%
%% Repeat \address commands as much as      %%
%% required.                                %%
%%                                          %%
%%%%%%%%%%%%%%%%%%%%%%%%%%%%%%%%%%%%%%%%%%%%%%

\address[id=aff1]{%
  \orgname{Centre de Recherche, Institut Universitaire de Gériatrie de Montréal},
  \city{Montréal},
  %
  %
  \postcode{Quebec},
  \cny{Canada}
}
\address[id=aff2]{%
  \orgname{Université de Montréal},
  \city{Montréal},
  %
  %
  \postcode{Quebec},
  \cny{Canada}
}
\address[id=aff3]{%
  \orgname{University of Massachusetts Medical School},
  \city{Worcester},
  %
  %
  \postcode{Massachusetts},
  \cny{USA}
}
\address[id=aff4]{%
  \orgname{University of California},
  \city{Berkeley},
  %
  %
  \postcode{California},
  \cny{USA}
}
\address[id=aff5]{%
  \orgname{Institut Pasteur},
  \city{Paris},
  %
  %
  %
  \cny{France}
}

%%%%%%%%%%%%%%%%%%%%%%%%%%%%%%%%%%%%%%%%%%%%%%
%%                                          %%
%% Enter short notes here                   %%
%%                                          %%
%% Short notes will be after addresses      %%
%% on first page.                           %%
%%                                          %%
%%%%%%%%%%%%%%%%%%%%%%%%%%%%%%%%%%%%%%%%%%%%%%

\begin{artnotes}
\end{artnotes}

%\end{fmbox}% comment this for two column layout

%%%%%%%%%%%%%%%%%%%%%%%%%%%%%%%%%%%%%%%%%%%%%%
%%                                          %%
%% The Abstract begins here                 %%
%%                                          %%
%% Please refer to the Instructions for     %%
%% authors on http://www.biomedcentral.com  %%
%% and include the section headings         %%
%% accordingly for your article type.       %%
%%                                          %%
%%%%%%%%%%%%%%%%%%%%%%%%%%%%%%%%%%%%%%%%%%%%%%

%\begin{abstractbox}

%\begin{abstract} % abstract
	
%Blank Abstract

%\end{abstract}



%%%%%%%%%%%%%%%%%%%%%%%%%%%%%%%%%%%%%%%%%%%%%%
%%                                          %%
%% The keywords begin here                  %%
%%                                          %%
%% Put each keyword in separate \kwd{}.     %%
%%                                          %%
%%%%%%%%%%%%%%%%%%%%%%%%%%%%%%%%%%%%%%%%%%%%%%

%\vskip1ex

%\projectURL{\url{http://github.com/r03ert0/brainspell-brainhack}}
%\projectURL{http://github.com/r03ert0/brainspell-brainhack}

% MSC classifications codes, if any
%\begin{keyword}[class=AMS]
%\kwd[Primary ]{}
%\kwd{}
%\kwd[; secondary ]{}
%\end{keyword}

%\end{abstractbox}
%
\end{fmbox}% uncomment this for twcolumn layout

\end{frontmatter}

%{\sffamily\bfseries\fontsize{10}{12}\selectfont Project URL: \url{http://github.com/r03ert0/brainspell-brainhack}}

%%% Import the body from pandoc formatted text
\section{Introduction}\label{introduction}

The past several decades have seen an explosive growth in the number of
published neuroimaging studies. In concert, the demand for freely
available and openly accessible `study data', that would facilitate
future reanalysis, meta-analysis, hypothesis testing and repurposing has
also soared. Here we report on developments made to
Brainspell\cite{brainspell}, one of several web-based initiatives
(e.g.~BrainMap\cite{fox1994}, NeuroVault\cite{neurovault},
Neurosynth\cite{neurosynth}) that allow individuals to search through
and organize massive numbers of neuroimaging studies and results in
meaningful ways.

Distinct from other databases, Brainspell (\url{http://brainspell.org})
is the first web-based initiative to allow users to manually annotate
and curate machine-parsed data, as well as manually extend the database
via its crowdsourcing user interface. The goal of our Brainhack project
was to improve Brainspell's interface. We worked to (a) provide
supplementary manual data edit options (b) facilitate efficient manual
database extension, and (c) aid meaningful organization of data.

\section{Approach}\label{approach}

We used GitHub to manage the client and server code, and to coordinate
its development.

\section{Results}\label{results}

\begin{figure}[h!]
  \includegraphics[width=.47\textwidth]{fig1}
  \caption{\label{centfig}A) 3D volume and mesh showing the aggregated locations of a user/peer-defined collection (Aman\_Metaanalysis) containing 32 articles. This user has a total of two collections (or 2 lists), as indicated on the header row. The second collection is named `test'. B) Highlighted in yellow are the \emph{Split} and \emph{Import} links associated with each table in Brainspell. Note: With the exception of the \emph{Download} link, peer-login is required to access all mentioned Brainspell enhancements.}
\end{figure}

\subsubsection{Supplementary manual data edit
options}\label{supplementary-manual-data-edit-options}

In the original version of Brainspell, users were able to edit
experiment (table) title, caption and coordinates for each article. We
added four supplementary options. In particular, users are now provided
with enhanced `edit feedback':

\begin{itemize}
\item Feedback indicating when a field is editable or has been successfully saved. Editable text fields now turn light grey, while a successfully stored field loses its coloring. Storage of fields can now be triggered by a tab key or by clicking elsewhere, in addition to hitting return.
\end{itemize}\vspace{1ex}

\noindent Users are also provided with additional edit options,
specifically, the ability to:

\begin{itemize}
\item Add symbols to the title and caption fields.
\item Remove empty tables.
\item Add and remove rows from a table. 
\end{itemize}

\subsubsection{Database extension}\label{database-extension}

While users were previously able to add new articles and their
coordinate tables, the process was labor- and time-intensive, since each
value had to manually entered. We implemented a more efficient method to
edit tables:

\begin{itemize}
\tightlist
\item
  Addition of an \emph{Import} link to each table. When clicked it opens
  a popup window where comma-separated text can be entered and parsed.
\end{itemize}

\subsubsection{Meaningful organization of
data}\label{meaningful-organization-of-data}

Potential shortcomings of neuroimaging databases employing automatic
coordinate data extraction is their inability to segregate (i) multiple
contrasts (e.g.~within group, inter-group), and (ii) significant versus
nonsignificant coordinates, when present in a single table. The
following options were added to facilitate non-ambiguous data
organization:

\begin{itemize}
\tightlist
\item
  Addition of a \emph{Split} link to each table.
\item
  Fine-tuning the \emph{Split} link enhancement to allow more than ten
  splits.
\item
  Option to add articles lacking PMID (or user-specific articles).
\item
  Addition of a \emph{Download} link to each article. When clicked it
  downloads article title, reference, abstract, and tables.
\item
  Creation of `article collection' functionality. Users can now store
  the results of their search into article collections. Clicking on an
  existing collection brings back the corresponding articles and
  re-computes the 3D volume and mesh of the aggregated locations. Users
  can create and edit multiple collections.
\end{itemize}

\section{Conclusion}\label{conclusion}

We performed ten enhancements to Brainspell and provided instructions of
use in Brainspell's wiki. We tested these enhancements on Safari,
Firefox and Chrome. Moreover, 25 articles were manually added to
Brainspell as part of our extended beta testing phase. Our goal with
these enhancements was to extend the functionality, and ease of use of
Brainspell for curating machine-parsed neuroimaging data from a wide
database of studies.

During January 15 to February 5, 2016 alone, Brainspell was used in 282
sessions by 133 users who watched 1421 pages. Moreover, Brainspell was
forked to ``BIDS-collaborative/Brainspell'' which itself was forked by
approximately 10 data-science students to extend the platform.

%%%%%%%%%%%%%%%%%%%%%%%%%%%%%%%%%%%%%%%%%%%%%%
%%                                          %%
%% Backmatter begins here                   %%
%%                                          %%
%%%%%%%%%%%%%%%%%%%%%%%%%%%%%%%%%%%%%%%%%%%%%%

\begin{backmatter}

\section*{Availability of Supporting Data}
More information about this project can be found at: \url{http://github.com/r03ert0/brainspell-brainhack}. Further data and files supporting this project are hosted in the \emph{GigaScience} repository REFXXX.

\section*{Competing interests}
None

\section*{Author's contributions}
RT developed Brainspell. AB, DK, and JBP suggested enhancements and
performed beta testing. AB and RT wrote the report.

\section*{Acknowledgements}
The authors would like to thank the organizers and attendees of
Brainhack 2015 OHBM Hackathon.

  
  
%%%%%%%%%%%%%%%%%%%%%%%%%%%%%%%%%%%%%%%%%%%%%%%%%%%%%%%%%%%%%
%%                  The Bibliography                       %%
%%                                                         %%
%%  Bmc_mathpys.bst  will be used to                       %%
%%  create a .BBL file for submission.                     %%
%%  After submission of the .TEX file,                     %%
%%  you will be prompted to submit your .BBL file.         %%
%%                                                         %%
%%                                                         %%
%%  Note that the displayed Bibliography will not          %%
%%  necessarily be rendered by Latex exactly as specified  %%
%%  in the online Instructions for Authors.                %%
%%                                                         %%
%%%%%%%%%%%%%%%%%%%%%%%%%%%%%%%%%%%%%%%%%%%%%%%%%%%%%%%%%%%%%

% if your bibliography is in bibtex format, use those commands:
\bibliographystyle{bmc-mathphys} % Style BST file
\bibliography{brainhack-report} % Bibliography file (usually '*.bib' )

\end{backmatter}
\end{document}
